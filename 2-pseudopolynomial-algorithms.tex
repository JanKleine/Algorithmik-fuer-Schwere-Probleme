\section{Pseudopolynomielle Algorithmen}

\paragraph{Zahlproblem (integer value problem IVP)} \mbox{} \\
Eingabe: darstellbar als Zahl $ x = x_1 \# \dots \# x_n \; ; \; x_i \in \binarystring $
und interpretiert als Vektor $Int(x) = (Number(x_i), \dots, Number(x_n))$. \\
Beispiel: Travelling Salesman Problem TSP (via Adjazenzmatrix des Graphen).

Sei $Max-Int(x) = \max \{ Number(x_i) \}$ die grösste vorkommende Zahl (im Wert, nicht in der Darstellung).
$Max-Int(x)$ kann exponentiell in $|x|$ sein.

\paragraph{Pseudopolynomiell}
Sei $U$ ein Zahlproblem und $\A$ ein Algorithmus der $U$ löst.
$\A$ heisst \emph{pseudopolynomiell} falls für alle Eingaben $x$ ein Polynom $p$ existiert, so dass
$$ \Time_\A (x) \in \bigO \Big( p \big( |x| , Max-Int(x) \big) \Big) $$
D.h. auf Eingaben mit ``kleinen Zahlen'' ist $\A$ polynomiell.

\paragraph{Rucksackproblem (Knapsack problem KP)} \mbox{} \\
Eingabe $I$: Gewichte $w_i \in \N^+$, Kosten/Nutzen $c_i \in \N^+$, Limit/Kapazität $b \in \N^+$, wo $i \in \{ 1, \dots, n \} $. \\
Ausgabe: Indexmenge $T \subseteq \{ 1, \dots, n \}$ s.t. $\sum_{i \in T } \leq b$ \\
Kosten: $cost(T, I) = \sum_{i \in T} c_i$ \\
Ziel: max

\underline{Lösung mit DP:}
Iteration über alle Teilprobleme $I_i$ und Speichern von Tripeln
$(k, W_{i, k}, T_{i, k})$ = (Nutzen, Gewicht, Indexmenge), also Mengen $T_{i, k}$
die exakt Nutzen $k$ mit minimalen Gewicht $W_{i, k}$ erreichen.
In jeder Iteration behalte für jeden vorhandenen Nutzen ein Tripel mit minimalen Gewicht.
Lese am Ende den maximal erreichten Nutzen $k^*$ (und sein $T_{n, k^*}$) aus.
\\
Laufzeit: $ \bigO \big( |I|^2 \cdot Max-Int(I) \big) \quad$ da $|I|$ Iterationen und
jeder Schritt in $\leq \sum_j^n c_j = |I| \cdot Max-Int(I)$.

