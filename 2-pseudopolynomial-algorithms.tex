\section{Pseudopolynomielle Algorithmen}

\begin{takeaway}
    \item Pseudopolynomialität
    \item Stark NP-schwer
\end{takeaway}

\paragraph{Zahlproblem (integer value problem IVP)} 
Probleme, dessen Eingaben vollständig durch Integer darstellbar sind.

Eingabe $x = x_1 \# \dots \# x_n \; ; \; x_i \in \binarystring$, interpretiert 
als Vektor $\text{int}(x) = (\text{number}(x_i), \dots, \text{number}(x_n))$.

Weiter sei 
\[\text{max-int}(x) = \max_{i \in \{1, ..., n\}} \text{number}(x_i)\]
die grösste vorkommende Zahl (im Wert, nicht in der Darstellung). Da die 
Eingabe binär definiert ist, kann $Max-Int(x)$ exponentiell in $|x|$ sein.

\textbf{Beispiel:} Travelling Salesperson Problem TSP (via Adjazenzmatrix des Graphen).

\paragraph{Pseudopolynomiell}
Sei $U$ ein Zahlproblem und $\A$ ein Algorithmus der $U$ löst.
$\A$ heisst \emph{pseudopolynomiell} falls für alle Eingaben $x$ ein 
Polynom $p$ existiert, so dass 
\[ \Time_\A (x) \in \bigO \Big( p \big( |x| , \text{max-int}(x) \big) \Big)\]
D.h. auf Eingaben mit ``kleinen Zahlen'' ist $\A$ polynomiell.

\paragraph{h-beschränktes Teilproblem}
Sei $U$ ein Zahlproblem, $h: \N \mapsto \N$ monoton nicht-fallend.
Das \emph{h-beschränkte Teilproblem von $U$ (h-value-bounded subproblem)} 
$\text{value}(h)\text{-}U$ ist das Teilproblem mit Eingaben $I$ für 
die gilt: $\text{max-int}(I) \leq h(|I|)$.

\paragraph{Stark NP-schwer}
Ein Zahlproblem $U$ heisst \emph{stark NP-schwer} falls ein Polynom $p$ existiert, 
so dass $\text{value}(p)\text{-}U$ NP-schwer ist.

In anderen Worten: $U$ ist stark NP-schwer wenn es auch dann NP-schwer ist wenn 
alle darin vorkommenden Zahlen ``klein'' sind.

Beispiel: TSP ist stark NP-schwer. Generell jedes gewichtete 
Graph-Optimierungsproblem ist NP-schwer, wenn das ungewichtete Pendant 
NP-schwer ist (bei TSP also das Hamiltonkreisproblem HCP).

\paragraph{Theorem}
Sei $U$ stark NP-schwer. Dann existiert kein pseudopolynomieller Algorithmus für $U$. 
\footnote{Wie immer unter der Annahme dass $P \neq NP$.}

Beweis Ansatz: Wir nehmen an, dass ein pseudo poly. Algorithmus $\mathcal{A}$ 
für $U$ existiert. Über die Definition von Stark NP-schwer können wir dann 
zeigen, dass wir ein polynomiellen Algorithmus für $\text{value}(p)\text{-}U$ 
hätten, was ein Widerspruch ist.


\subsection{Rucksackproblem (Knapsack problem, KP)}\label{algo-kp}

\paragraph{Rucksackproblem}
Wie viel Diebesgut kann ich mitnehmen.

\textbf{Eingabe $I$:} Gewichte $w_i \in \N^+$, Kosten/Nutzen $c_i \in \N^+$, 
Limit/Kapazität $b \in \N^+$, wo $i \in \{ 1, \dots, n \} $. \\
\textbf{Ausgabe:} Indexmenge $T \subseteq \{ 1, \dots, n \}$ s.t. $\sum_{i \in T } \leq b$ \\
\textbf{Kosten:} $cost(T, I) = \sum_{i \in T} c_i$ \\
\textbf{Ziel:} $\max$

\paragraph{Lösung mit Dynamischer Programmierung (Dyn-Prog-KP)} 
Wir iterieren über alle Teilprobleme $I_i$ und speichern die Tripel
$(k, W_{i, k}, T_{i, k})$ = (Nutzen, Gewicht, Indexmenge), also Mengen $T_{i, k}$
die exakt Nutzen $k$ mit minimalen Gewicht $W_{i, k}$ erreichen.
In jeder Iteration behalte für jeden vorhandenen Nutzen ein Tripel mit minimalen Gewicht.
Lese am Ende den maximal erreichten Nutzen $k^*$ (und sein $T_{n, k^*}$) aus.

\paragraph{Laufzeit} \[\bigO \big( |I|^2 \cdot \text{max-int}(I) \big)\] da der 
Algorithmus $|I|$ Iterationen durchläuft und
jeder Schritt in $\leq \sum_j^n c_j = |I| \cdot \text{max-int}(I)$ läuft.

\paragraph{Verbesserung falls \(b \ll \sum_{j=i}^{n} c_j\):}
Wir können den Algorithmus für Instanzen verbessern, 
bei denen die Gewichtsschranke kleiner ist als die Summe aller Nutzen.

Im oben stehenden Algorithmus speichern wir uns ein Tripel für jeden möglichen 
Nutzen (es gibt \(\sum_{j=i}^{n} c_j\) solche Werte) das minimal Gewicht (und 
die jeweiligen Gegenstände), das diesen Nutzen erreicht. Da \(b \ll \sum_{j=i}^{n} c_j\) 
können wir die Logik des Algorithmus ``umkehren'' und stattdessen für jedes 
mögliche Gewicht den maximal erreichbaren Nutzen speichern. 

Die neue Laufzeit des neuen Algorithmus ist in \(\bigO \left (n \cdot b\right)\), 
wobei \(n \in \bigO(|I|)\) und \(b \in \bigO(\text{max-int}(I))\).

\paragraph{Weitere Verbesserung} Wir können die zwei Ansätze kombinieren für 
Probleme bei denen \(b \gg \sum_{j=i}^{n} c_j\), wobei allerdings nur wenig 
Objekte die Summe stark beeinflussen. Also wir nehmen an es gibt ein \(i \ll n\), 
sodass \(b \gg \sum_{j=i}^{n - i} c_j\) und \(b \ll \sum_{j=i}^{n} c_j\).

Hierfür unterteilen wir das Problem in zwei Teilprobleme, eines, was nur die 
``leichten'' Objekte enthält und eines welches nur die ``schwere'' Objekte beinhaltet. 
Wir rufen den ursprünglichen Algorithmus auf der ``leichten Instanz''\footnote{Leichte 
Instanz im Sinne von Gewicht, nicht Komplexität} auf und die oben beschriebene 
Verbesserung auf die ``schwere Instanz''. Wir können die beiden Lösungen nun kombinieren, 
indem wir für jede Lösung des einen Problems, die best mögliche Lösung aus dem anderen 
Problem suchen. Die Lösung ist die Kombination mit dem höchsten Gesamtnutzen.

Die Laufzeit des Algorithmus ist in
\[\bigO\left(n \cdot \sum_{j = 1}^{n-i} + i \cdot b \right).\]
